\documentclass[a4paper]{article}

\usepackage[latin1]{inputenc}
\usepackage[english]{babel}
%\usepackage{textcomp}
\usepackage[T1]{fontenc}
\usepackage{type1ec}%per i font cmsuper
\usepackage[pdftex]{graphicx}
\usepackage{epstopdf}
\usepackage[bw,framed]{mcode}
%\usepackage{titlepage}
%\usepackage{fancyhdr}
%\usepackage{fancybox}
\usepackage{color}
\usepackage{url}
\usepackage{appendix}
\usepackage{hyperref}
\usepackage{makeidx}
\usepackage{amssymb}
\usepackage{amsfonts}
\usepackage{amsmath}
\usepackage{mathtools}
\usepackage{theorem}
\usepackage{lscape}
\usepackage{glossary}
\usepackage{array}
\usepackage{xtab}
\usepackage{tabularx}
\usepackage{eufrak} % Per fare la F particolare!
\usepackage{tocloft}
\usepackage{psfrag}
\usepackage{subfigure}
\usepackage{verbatim}
%\usepackage[left=2.5cm,top=2.5cm,right=2.5cm,bottom=2.5cm,nohead]{geometry}
\usepackage{placeins}
%\usepackage{a4wide}
\usepackage{epsfig}
\usepackage{psfrag}
\usepackage[strict]{chngpage}

\newcommand{\addappendixname}{%
  \renewcommand{\cftchappresnum}{\appendixname\space}%
    \settowidth{\cftchapnumwidth}{\bfseries \appendixname\space A\hspace{1em}}}

\title{\texttt{LifeV} Development Guidelines}
\author{C. Malossi, S. Deparis}
\date{\today}

\begin{document}

\maketitle

This short guide provides some useful rules to improve the
development of \texttt{LifeV} Library. Following the example of
\texttt{Trilinos} and other well structured libraries, we have understood that
it is mandatory to follow some common rules in the implementation of our code. 
A better organization of classes and files, together with the development of a
complete and exaustive documentation, will help old and new developers to
take advantage of all the available features of our library, improving
productivity and results.

\section{Nomenclature of classes and methods}
This is a short list of rules for the nomenclature of files, classes and
methods.
\begin{itemize}
  \item All declarations should be placed into \texttt{*.hpp} files. Method
  definitions (except inlined methods or other possible special cases)
  should be inside the \texttt{*.cpp} files.
  \item Use type definitions at the beginning of the class to introduce more
  generality in the code. The syntax for types and pointer types is
  the following:
\begin{lstlisting}
typedef Epetra_FEVector                       vector_Type;
typedef const Epetra_FEVector                 vector_ConstType;

typedef boost::shared_ptr< vector_Type >      vector_ptrType;
typedef boost::shared_ptr< vector_constType > vector_ptrConstType;
\end{lstlisting}
  \item Classes, structures and files names should start with the capital
  letter.
\begin{lstlisting}
class datatime; //NO!!!
  
class DataTime; //OK
\end{lstlisting}
  \item In case of composed names, each word (starting from the second one)
  should start with capital letter.
\begin{lstlisting}
void matrixvectormultiplication( matrixtype& matrix ); //NO!!!
  
void matrixVectorMultiplication( matrix_Type& matrix ); //OK
\end{lstlisting}
  \item All variables and methods names should clearly describe what they are
  doing.
\begin{lstlisting}
// This is a true bad example taken from EpetraMatrix class
  
// What do these mean?
void set_mat_inc( UInt row, UInt col, DataType loc_val );
void set_mat( UInt row, UInt col, DataType loc_val );
  
// Now it is clear!
void insertSingleElement( UInt row, UInt column, data_Type value );
void replaceSingleElement( UInt row, UInt column, data_Type value );
  
  \end{lstlisting}
  \item Do not use abbreviations in the code! They prevent its readability.
\begin{lstlisting}
void elByElMul( vtype& v1, vtype& v2 ); //NOOOOO!!!
  
void elementByElementMultiplication( vector_Type& vector1,
                                     vector_Type& vector2, ); //OK
\end{lstlisting}
  \item Member variables in \texttt{LifeV} should start with \texttt{M\_}.
  Moreover all the rules for the names of methods are also valid for class
  members. 
\begin{lstlisting}
private:

    Real        vel;   //NO!!!
    std::string File;  //NO!!!
    
    Real        M_velocity; //OK
    std::string M_fileName; //OK
\end{lstlisting}
  \item Static members variables in \texttt{LifeV} should start with
  \texttt{S\_}. Moreover all the rules for the names of methods are also valid for class
  members. 
\begin{lstlisting}
private:

    static Int counter;   //NO!!!
    static Int M_counter; //NO!!!
    
    static Int S_counter; //OK
\end{lstlisting}
  \item Avoid function declarations without argument name.
\begin{lstlisting}
void setDataFile( const std::string& ); //NO!!!

void setDataFile( const std::string& fileName ); //OK
\end{lstlisting}
\newpage
  \item All get functions and methods should be declared const, and should
  return a const object.
\begin{lstlisting}
matrix_Type&   matrix();     //NO!!!
matrix_ptrType matrix_ptr(); //NO!!!
  
const matrix_Type&   matrix() const;     //OK
const matrix_ptrType matrix_ptr() const; //OK
\end{lstlisting}
  \item All set functions and methods should start with the prefix
  \texttt{set}.
\begin{lstlisting}
void matrix( matrix_Type& matrix );          //NO!!!
void matrix_ptr( matrix_ptrType matrix_ptr); //NO!!!
  
void setMatrix( matrix_Type& matrix );          //OK
void setMatrix_ptr( matrix_ptrType matrix_ptr); //OK
\end{lstlisting}
  \item All principal classes should have a method \texttt{showMe()},
\begin{lstlisting}
void showMe( std::ostream& output = std::cout ) const;
\end{lstlisting}
  which outputs on \texttt{output} (which may be the standard output
  or a file) the state of the class (i.e.  the content of the global
  and local variables).
\end{itemize}

\newpage
\section{Conventions}
This is a short list of programming and general conventions that should be used
working with \texttt{LifeV} library.

\subsection{Programming conventions}
\begin{itemize}
  \item Use \texttt{boost::shared\_ptr} instead raw pointers. If the raw
  pointer is needed by another external class (for example by \texttt{Trilinos})
  you can use the \texttt{get()} method of \texttt{boost::shared\_ptr} class.
  \item Don't use reference types, expecially for members inside classes.
  \item \texttt{using} directives should be avoided entirely, especially in
  header files. They cause wanton namespace pollution 
  by bringing in potentially huge numbers of names, many (usually the vast
  majority) of which are unnecessary. The presence of the unnecessary names
  greatly increases the possibility of unintended name conflicts not just in
  the header, but in every module that includes the header. Moreover
  \texttt{using namespace} declarations should never appear in header files.
  The meaning of the \texttt{using} declaration may change depending on what
  other headers have included before it in any given module.
  \item Use \texttt{const} keyword when possible. This helps the compiler
  and the developers reading the code. Moreover, it enhances
  debugging. See \url{http://en.wikipedia.org/wiki/Const-correctness} for more
  details on the use of \texttt{const}.
  \item Use the \texttt{C++} cast utility (\texttt{static\_cast}) instead of
  implicit compiling casts. This will avoid warnings and will keep
  each cast visible for future debugging. See
  \url{http://www.acm.org/crossroads/xrds3-1/ovp3-1.html} for more details on
  the use of cast commands.
  \item Use the typedefs aliases \texttt{Real}, \texttt{Int} and
  \texttt{Uint}, instead of the built-in types \texttt{double},
  \texttt{int} and \texttt{unsigned int}. This helps making code
  changes afterwards. All the type definitions are contained into
  \texttt{life.hpp}. \newline \textbf{NOTE:} Only for MPI instructions and
  Trilinos call functions, it could be necessary to use \texttt{int} instead of \texttt{Int}.
  In these cases, use \texttt{static\_cast} to avoid warning messages.
  \item For debugging purposes please use the \texttt{debug.hpp} class features:
\begin{lstlisting}
#ifdef DEBUG
    Debug( 3000 ) << "MyClass::myFunction  myVariable = " 
                  << M_myVariable << "\n"; 
#endif
\end{lstlisting}
  Thanks to this syntax no output will be displayed, unless the debug
  mode is enabled and an enviroment variable is set in the shell with the
  command \texttt{export DEBUG="3000"}. This will avoid a lot of unuseful
  outputs especially on parallel runs. For the complete list of debug numbers
  (that you can enrich) see the file \texttt{debug.areas} in the library.
  \item For assertion purposes please use the \texttt{lifeassert.hpp} class
  features.
\end{itemize}

\subsection{General conventions}
\begin{itemize}
  \item To uniform the code inside the library, all developers have to
  use the same notation and in particular the same indentation style. The style
  used in \texttt{LifeV} is the BSD/Allman Indent style. Please see
  \url{http://en.wikipedia.org/wiki/Indent_style#Allman_style_.28bsd_in_Emacs.29} 
  for more details about this style and its inventor.
  Moreover please note that both Eclipse and Emacs (and probably also other editors)
  have a special utility to help follow this style.
  \item Before committing any code, be sure that all tabs have been converted
  into $4$ spaces, and that your editor has removed endline spaces.
  \item Before committing any modification in the library, always check
  that these modifications have not broken any test present in the
  testsuite.
  \item Always include a complete description of all the modifications in the
  commit message, to allow other developers to easily understand what is
  changing in the library.
  \item About the debugging procedure: if you find a bug in the code, fix it
  and then use the \texttt{lifev-dev} mailing list to notify all the developers
  about the bug. A simple commit with short description is not sufficient!
  \item All the documentation, files and variables names, comments and more
  generally any kind of text must be in english language.
  \item Don't write comment or documentation using all caps look letters. Use
  them only for titles, or specific words.
  \item In order to improve readability of the code, headers should
  contain sections (see Section \ref{doxygen} for doxygen syntax) grouping
  all the similar methods. Basically all the classes should contain at least
  these sections:
  \begin{description}
      \item[Public Types]: containing the public enum and type definition(s).
	  \item[Constructors \& Destructor]: containing the constructor(s), the
      copy constructor and the destructor(s).
      \item[Operators]: containing the operators defined in the class, such as
      the \texttt{operator=} for making copies.
      \item[Methods]: containing all the general methods of the class. Note that
      a method performs operations on private variables and it is
      not just a setter or getter function.
      \item[Set Methods]: containing all the set methods (starting with the
      prefix \texttt{set})
      \item[Get Methods]: containing all the get methods.
      % starting with the prefix \texttt{get})
  \end{description}
\end{itemize}

\newpage
\section{Testsuite development} \label{testsuite}
All new packages and main classes should have a working test in the testsuite:
this is mandatory to allow easy mantainance of all the classes in the
library. Moreover, it is important to have a working test for the night
compilation of the library, that runs all the tests in order to verify their
status.

\subsection{How to add a test in the testsuite}
The first step to add a new test in the testsuite is the creation of a new
folder, with the name \texttt{test\_NameOfTheTest}. Inside the folder, at least
these files should be present:
\begin{description}
	\item[main.cpp]: the main file for the test, containing information about the
	test and a doxygen description of its purposes. A template for this file is
	provided in appendix \ref{main}.
	\item[Makefile.am]: the makefile of the test. A template for this file is
	provided in appendix \ref{Makefileam}. \textbf{NOTE:} To enable the automatic
	compilation of the test, it is necessary to add a line in the
	\texttt{Makefile.am} file placed in the main testsuite folder.
	\item[data.txt]: the data file(s) for the test, which should be a significant
	case. This data file is used when a manual execution of the test is performed.
	\item[testsuite.at]: the configure file for the night test, which could
	contain the same test as the one in the \texttt{data.txt}, or a different one.
	A template for this file is provided in appendix \ref{testsuiteat}.
	\textbf{NOTE:} To enable the night execution of the test, it is necessary to
	add a line inside the \texttt{Makefile.am} and \texttt{testsuite.at}
	files placed in the main testsuite folder.
\end{description}
Note that \texttt{data.txt} and \texttt{testsuite.at} are independent.
The test should verify both the compilation and execution of the test. For this
second task, it is necessary to place a check at the end of the main file. The
check could consist in a tolerance test (for a numeric comparison of the
obtained result with the expected one), or a flag check. Here there is an
example:
\begin{lstlisting}
Real tolerance = 1.e-10;
if ( tolerance > result )
    return EXIT_FAILURE;
else
    return EXIT_SUCCESS;
\end{lstlisting}
Please also note that the testsuite is not the right place for applications.
The testsuite has the purpose to test classes and packages with simple tests.
Moreover, the tests should be no longer than $5$
minutes, to allow a quick check of the library before committing new software.

\newpage
\section{Doxygen documentation} \label{doxygen}
All classes should have documentation lines explaining in a concise
but thorough way their usage. Doxygen provides an easy and general format to
obtain this result.

\subsection{How to obtain \texttt{LifeV} doxygen documentation}
If the \texttt{doxygen} and \texttt{graphviz} packages are
present on your machine, \texttt{LifeV} automatically generates doxygen
documentation at the end of each compilation process. The documentation is
accessible at the following path: \newline

\texttt{\$LIFEV-COMPILATION-FOLDER/doc/api/html/index.html}\newline

\subsection{How to add doxygen style in \texttt{C++} classes}
In order to be fully documented, the code should follow some simple rules which
are described on the doxygen website \url{www.doxygen.org}. In particular the
main doxygen style commands are described at the following page
\url{http://www.stack.nl/~dimitri/doxygen/manual.html}. 

To make things easier for all \texttt{LifeV} developers, a general
\texttt{LifeV} template class has been generated and attached to this guide
as an appendix (see appendix \ref{TemplateClassHPP} and
\ref{TemplateClassCPP}). It contains the general layout structure for any new
class. All the future classes in \texttt{LifeV} should be developed starting
from these two files, which contain everything to make doxygen work
properly.

\bigskip
\noindent In particular, the header file is divided into $4$ sections:
\begin{enumerate}
  \item A general header containing license and copyright information that
  should appear at the beginning of each file.
  \item A short description of the file content, with at least a brief description of the file
  content (using command \texttt{@brief}), the author(s) list (using
  \texttt{@author}) and the date (using
  \texttt{@date}). In particular, for the description of the file it
  could simply be the list of classes contained in the file, which are fully
  described later.
  \item \label{point3} A full description of the class,
  describing its purpose and its external interface. It should be
  placed before its declaration. The first line (starting with
  \texttt{//!}) contains the name of the class and a very short
  description of it (one line maximum). In the following block of lines
  (starting with \texttt{/*!} and finishing with \texttt{/*}) the
  full description of the class should be given. This is a good place also to
  describe the list of parameters that are required by the class and that
  should be provided from a data file.
  \newpage
  \item \label{point4} A detailed description of all the public methods
  contained in the class, and their input and output
  parameters. All the methods are grouped using the syntax:
  \begin{lstlisting}
//! @name Name of the group (for example Methods)
//@{

//@}
\end{lstlisting}
  and inside each group all the methods contain a full description of their
  usage such as:
\begin{lstlisting} 
//! Short description of the equivalence operator
/*!
   Add more details about this operator.
   NOTE: short description is automatically added before this part.
   @param T  TemplateClass
   @return   Reference to a new TemplateClass with the same
             content of TemplateClass T
 */
TemplateClass& operator=( const TemplateClass& T );
\end{lstlisting}
  where all the parameters are described using the syntax:
\begin{lstlisting}
@param inputParameter   Description of the input parameter
@return                 Description of the output parameter
\end{lstlisting}
 Note that for input parameters (command \texttt{@param}) the first word is the
 name of the parameter, followed by his description, while for output
 parameters (command \texttt{@return}) there is only the description.
\end{enumerate}
If there are more classes (or structures) in the file, point \ref{point3} and
\ref{point4} should be repeated for all of them.

For the \texttt{.cpp} file the first two sections are exactly the same of the
\texttt{*.hpp} file. Then all the implementations of the methods should be
placed in the same order as in the header file. To keep the same group division
given in the header, the following comment could be used to identify the
beginning of a group:
\begin{lstlisting}
// ===================================================
// Name of the group
// ===================================================
\end{lstlisting}

\appendix
\addtocontents{toc}{\protect\addappendixname}
\clearpage

\changepage{20em}{20em}{0em}{-10em}{0em}{-10em}{}{}{}

\section{TemplateClass.hpp - Template Class header} \label{TemplateClassHPP}
\small{\lstinputlisting{TemplateClass.hpp}}

\clearpage
\section{TemplateClass.cpp - Template Class sourse code}
\label{TemplateClassCPP}
\small{\lstinputlisting{TemplateClass.cpp}}

\clearpage
\section{main.cpp - Testsuite main file} \label{main}
\small{\lstinputlisting{main.cpp}}

\clearpage
\section{Makefile.am - Testsuite make file} \label{Makefileam}
\small{\lstinputlisting{LifeV_Makefile.am}}

\section{testsuite.at - Testsuite night test file} \label{testsuiteat}
\small{\lstinputlisting{testsuite.at}}


\end{document}
