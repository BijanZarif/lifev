\documentclass[a4paper,10pt]{article}

\usepackage{amsmath}
\usepackage{amsfonts}
\usepackage{amssymb}
\usepackage{verbatim}
\usepackage[english]{babel}
\usepackage{epsfig}

% writing layout
\setlength{\textwidth}{16.5cm}
\setlength{\textheight}{21.0cm}
\setlength{\unitlength}{1pt}
\setlength{\oddsidemargin}{-0.5cm}
\setlength{\topmargin}{-1cm}
% \setlength{\topmargin}{-1.5in}
\setlength{\parskip}{2ex}

\newcommand{\pst}[1]{\texttt{#1}}

\newcounter{rem}
\newenvironment{rem}{\noindent {\bf Remark~\refstepcounter{rem}\therem~--}\ }{{\footnotesize $\blacksquare$} \par}

%%%%%%%%%%%%%%%%%%%%%%%%%%%%%%%%%%%%%%%%%%%%%%%%%%%%%%%%%%%%%%%%%%%%%

\title{Commenting the C++ library LifeV with DOXYGEN}

\author{Alain Gauthier}

\begin{document}

\maketitle

The program development documentation rules of the library LifeV have been
defined in the reference document. Here an indication is given on how
those commenting rules could be followed under DOXYGEN framework. This
framework can be easily adopted from any already commented peace of code.
Developing a DOXYGEN documentation consists indeed in placing keywords
in the code comments which are interpreted as DOXYGEN commands.

Instead of using DOXYGEN in a sophisticated way, it could be suitable
to use it in a simple systematic way in order to insure that all parts
of the library are well commented. So only the simplest way to generate
the library documentation is presented here.

In the following documentations are presented for files, classes, and
functions.

\section{File documentation}

In any file (.h or .cc) of the library, a comment block at the beginning
of the file should contain the following information (for a file named
``filename''):

\begin{verbatim}
/*!
  \file filename

  \brief brief description of the file purpose.

  \version 1.0 (might be given by CVS...) Date

  \author your name.

  More detailed description of the file purpose (if necessary).

  \todo remarks reminding important things to do.

*/
\end{verbatim}

\section{Class documentation}

All members of a class are documented as well as methods and friend
functions. The comments are done like in a standard situation except that
key characters like ``!'' or ``!$<$'' are employed just after the beginning
of a comment line or a comment block. The second key character refers to
the description of a class member which is preceding instead of the first key
character which describe a class member which is following.

There are different possibilities to write comments at the
level of class declaration as shown in the following example:

\begin{verbatim}
//! A test class.
/*!
  A more elaborate class description.

  \todo an eventual comment on work which has to be done...
*/

class Test:
{
 public:
  typedef unsigned int UInt; //!< description of the typedef...

  //! A constructor.
  //! few details..
  Test();

  //! A destructor.
  /*!
    A more elaborate destructor description...
  */
  ~Test();

  //! A normal member description
  /*!
    \param a1 the first argument, an integer.
    \param a2 the second argument, a real.
    \return the eventual member result.
  */
  void testMe(int a1, real a2);

 private:
  ObjetType _objet; //!< description of the private member...
};
\end{verbatim}

\section{Function documentation}

The principle is the same as for class member function documentation.
Keywords like \texttt{$\backslash$param} and \texttt{$\backslash$return} allow
to enhance the role of inputs and outputs arguments.
A function declaration documentation can be structured as following:

\begin{verbatim}
/*!
  \brief fun description

  A more elaborate fun description.

  \param a1 first argument description
  \param a2 second argument description
  ...
  \param an n-th argument description

  \return description of eventual fun output

  \todo an eventual comment on work which has to be done...
*/
TypeOutput fun(type1 a1, type2 a2, ...,type n);
\end{verbatim}

\section{Examples in the library LifeV}

Two examples taken
from the library LifeV (files algebraic\_facto.h and bareitems.h) of
{\it html} documentation generated using DOXYGEN are presented in the
following pages.

\end{document}
