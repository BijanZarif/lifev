
\chapter{Generalities}
\label{cha:generalities}

\section{Scope of the document}
\label{sec:scope-document}

This is an informal document dedicated to amateur or inexperienced users of the
software library \thelibrary (life 5).

The major objectives of this document are:
\begin{enumerate}
\item to compile the software library,
\item to provide examples of its use.
\end{enumerate}
For a more detailed overview of \lifev's main features (management of boundary conditions, time and space discretization, algebraic solvers and preconditioners, etc), see the doxygen webpage: \url{http://cmcsforge.epfl.ch/www/lifev/doxygen-release-2.2.0/lifev/}.
\section{Language and nomenclature convention}
\label{sec:lang-nomencl-conv}

\texttt{typesetting font style} is used to indicate parts of
computer code, configure shell scripts, command-prompt instructions and webpages.

\section{Software Management}
\label{sec:software-management}

The software source, its documentation and all related documents (this
one included) are kept in a repository under revision control
using \verb!git!\ixns{git}{versioning}\footnote{git is the fast version control system.}. Its goal
is to provide tools to manage software development in a concurrent environment.
See \url{http://git-scm.com/documentation}
for a tutorial.

As mentioned above, a website {\it\`a la} Sourceforge\footnote{\url{http://www.sourceforge.net}}
\url{http://cmcsforge.epfl.ch} has been set up
to host the source code and help the software management.
It requires that you open an account\footnote{\url{https://cmcsforge.epfl.ch/account/register.php}}
there and ask to join the project \lifev using the link at the
bottom of the developers' list.
Once you would become a member, you will gain access to all the facilities:
tracker, task manager, git repository, forums, document manager and a few other
tools which are very useful if not absolutely essential to such a project.

Finally, if you expect a frequent use of the \verb!git!\ixns{git}{versioning} repository
we recommend to costumize the
\verb!ssh! and \verb!ssh-agent!
 in order to gain acces  without the need to type your password everytime you issue
a command. Please refer to \url{http://mah.everybody.org/docs/ssh} in order to configure
your ssh agent.

We advice every user to apply to the list lifev-users on \url{http://groups.google.com}
where one can get in touch with other users and developers.

\section{Compiling LifeV}
\label{compile-lifev}

There are a few compilation tools and libraries we need to build and install before
compiling \lifev, here is a short presentation. Note that, in addition to the following description, the complete installation steps are available on the following webpages:
\begin{enumerate}
\item \url{http://www.lifev.org/documentation/installation-tutorial}\,,
\item \url{http://www.lifev.org/documentation/lifev-on-macosx}\,.
\end{enumerate}



In computer science, a library is a set of subroutines or classes used to develop software.
Usually they are downloaded as a so called ``tarball" file
compressed using the \verb!tar! command. There are different ways to compress
libraries but the most common is to use the command \verb!tar -cvf! and further compress
``zip" it with \verb!gzip!. If your tarball has the suffix \verb!.tar.gz! equivalent to \verb!.tgz!,
you can decompress ``unzip" it with \verb!gunzip! followed by the name of the \verb!.tar.gz! file and
extract its contents using \verb!tar -xvf! followed by the name of the \verb!.tar! file.
If you find the libraries compressed with other formats
please refer to the unix manuals \verb!man! or the numerous
on-line documents for further information.

Software libraries need to be extracted, compiled and installed. In unix-like
systems, the libraries \verb!.a! and
\verb!.so! files are installed usually in the directory \verb!/usr/lib!,
 while header files
\verb!.h! are installed in the \verb!/usr/include! directory. Compilers search
for libraries there by default, but in principle they can be installed anywhere you want
as long as you pass the path to the library using the compiler flag \verb!-L! immediately
followed by the library path and similarly for the header files using the compiler
flag \verb!-I! followed by the include path.

Libraries are usually created with the prefix \verb!lib! followed by the name of the library
and linked with the compiler flag \verb!-l! followed by its name.

\subsubsection{Compilation Environment}
\label{sec:comp-envir}

\lifev depends on a number of tools at compilation time that are part
of the \ix{autotools} from the GNU project\footnote{\url{http://www.gnu.org}}
available in most Linux OS:

\begin{itemize}
\item \verb!g++-4.0!\ixns{g++}{compilers} or newer (currently $4.5.4$)%$4.4.4$)
\item \verb!mpi!\ixns{mpi}{compilers}, with preference to \verb!openmpi!\ixns{openmpi}{compilers}
\item \verb!CMake 2.8! or newer.
\end{itemize}

In Mac OS X you get gcc in Xcode and cmake can be
installed using MacPorts \verb!sudo port install cmake!. You can check the version of a command typing the command followed by \verb!--help!,
for example type \verb!cmake --help!.

\lifev depends on several optimized libraries, you can check if you have them installed
using the \verb!locate! command followed by the name of the library, for example
\verb!locate liblapack.a!, or go to the \verb!/usr/lib! directory and search on the
list with \verb!ls!.
It is important to notice that some libraries are linked to others and they should
be compatible, therefore you should build them in the order of dependency and with
compatible flags and compilers.

These are the optimized libraries you need to have installed:

\begin{itemize}
\item A version of \verb!MPI!.
The message passing interface for C and Fortran compilers. For example \url{http://www.open-mpi.org/}.
Once installed you can check the necessary flags for its use by typing
\verb!mpicc --show!.
In Mac OS X using MacPorts install a fortran compiler typing
\verb!sudo port install gcc46! and openmpi with \verb!sudo port install openmpi!. Note however that MPI should be natively installed if you installed XCode.

\item \verb!BOOST!.
Libraries to extend the functionality of C$++$. Check if they exist on your computer, they are many
libraries with the prefix \verb!libboost!.
If you need to install them, try \verb!sudo apt-get install libboost*! in devian systems or
something similar in others.
If you can't do something like that then download at \url{http://www.boost.org}.
Make sure you include the line ``\verb!using mpi;!" in the configuration text file \verb!project-config.jam!.
You can specify the path to install using the flag \verb!--prefix=/path/! when running \verb!./bjam!
\verb!install!. But most of the time cross compilation of this library won't work completely.
In Mac OS X using MacPorts type \verb!sudo port install boost!.

\item \verb!HDF5!
If you don't have the library hdf5 installed in your system, you could use the
\verb!sudo! \verb!apt-get install libhdf5-openmpi-dev! instruction in devian linux systems or something similar
for your particular system. There are detailed instructions on-line on how to build it for other systems and
with other options, for example at \url{http://micro.stanford.edu/wiki/Install_HDF5#Build_and_Installation_from_Sources}.
In Mac OS X using MacPorts type \verb!sudo port install hdf5! or build it from the sources to link it to the correct
openmpi compilers.

\item \verb!BLAS!.
For example get \url{http://www.tacc.utexas.edu/tacc-projects/gotoblas2/}. To build just type \verb!make!.
To make use of the library remember to
have the pthreads library and flag  \verb!-lpthread! while linking to the blas library  \verb!libgoto2_xxxxx_xx.xx.a!,
whose exact name depends on the characteristics of your hardware. In Mac OS X
the system comes with blas and lapack as part of the Accelerate framework \verb!-framework Accelerate!,
and if using MacPorts type
\verb!sudo port install atlas! to install the atlas library (blas and lapack).

\item \verb!LAPACK!.
Fortran $90$ Linear Algebra Routines for systems of simultaneous linear algebra equations, linear least-squares problems
and matrix eigenvalue problems. You must pay attention to build the lapack using an optimized blas like the gotoblas of
last item. Download at \url{http://www.netlib.org/lapack/}. You need a fortran compiler (for example gfortran).
Copy \verb!make.inc.example! to \verb!make.inc! and edit the path to the blas library followed by the flag \verb!-lpthread!
and type \verb!make!.

\item \verb!PARMETIS!.
You can download it from \url{http://glaros.dtc.umn.edu/gkhome/metis/parmetis/download}. Set \verb!CC=mpicc!
in \verb!Makefile.in!. and type \verb!make!. In Mac OS X you need the include path flags
\verb!-I/usr/include! and \verb!-I/usr/include/malloc!.

\item \verb!UMFPACK!.
Set of routines for solving unsymmetric sparse linear systems.
Download from \url{http://www.cise.ufl.edu/research/sparse/umfpack/}. Requires
the packages \verb!UFconfig! and \verb!AMD!. Place amd, ufconfig and umfpack in the same parent directory.
Modify the file \verb!UFconfig.mk! to specify the C compiler \verb!CC = gcc!, \verb!CPLUSPLUS=g++!,
\verb!CFLAGS=-O3!
\verb!-fexceptions!
and the blas, lapack and metis paths and libraries including the \verb!-lpthread! flag.
For Mac OS X you must uncheck the special options given for this system, you can use the blas and lapack
from atlas or from the Accelerate framework \verb!-framework Accelerate!.
Type \verb!make! in
the umfpack directory. Move the files in amd \verb!Lib! and \verb!Include! to the umfpack \verb!Lib! and \verb!Include! directories.
Finally move \verb!UFconfig.h! to the corresponding \verb!Include! directory of umfpack.

\item \verb!TRILINOS!\ixns{aztec}{algebra}.

\end{itemize}

%\subsubsection{Runtime Environment}
%\label{sec:runtime-env}

%\begin{itemize}
%\item \verb!aztec!
%\end{itemize}

\subsection{Trilinos compilation}
\lifev depends on Trilinos, a set of object oriented C$++$ interfaces for packages like
blas, lapack, parmetis, umfpack and many more. A copy of the source
code is available for download at \url{http://trilinos.sandia.gov/packages/}.

After downloading, decompressing and extracting the tarball, you'll need to
make a build directory anywhere you want to avoid build in the sources directory.
The autotools build is no longer available after the $10.0$ release.
Trilinos now requires the CMake
build system, version $2.8$ or newer.
Go to the build directory and write a do-configure shell script like the following


\begin{verbatim}
#!/bin/bash

EXTRA_ARGS=$@

cmake \
    -D CMAKE_BUILD_TYPE:STRING=RELEASE \
    -D Trilinos_ENABLE_TESTS:BOOL=OFF \
    -D Trilinos_ENABLE_Epetra:BOOL=ON \
    -D Trilinos_ENABLE_Tpetra:BOOL=ON \
    -D Trilinos_ENABLE_Kokkos:BOOL=ON \
    -D Trilinos_ENABLE_EpetraExt:BOOL=ON \
    -D Trilinos_ENABLE_AztecOO:BOOL=ON \
    -D Trilinos_ENABLE_Amesos:BOOL=ON \
    -D Trilinos_ENABLE_Teuchos:BOOL=ON \
    -D Trilinos_ENABLE_Ifpack:BOOL=ON \
    -D Trilinos_ENABLE_ThreadPool:BOOL=ON \
    -D Trilinos_ENABLE_ML:BOOL=ON \
    -D Trilinos_ENABLE_Triutils:BOOL=ON \
    -D Trilinos_ENABLE_Zoltan:BOOL=ON \
    -D Trilinos_ENABLE_Galeri:BOOL=OFF \
    -D Trilinos_ENABLE_Isorropia:BOOL=OFF \
    -D Trilinos_EXTRA_LINK_FLAGS:STRING="-lpthread" \
    -D TPL_ENABLE_UMFPACK:BOOL=ON \
    -D UMFPACK_LIBRARY_DIRS:PATH=/umfpack_library_path/ \
    -D UMFPACK_INCLUDE_DIRS:PATH=/umfpack_include_path/ \
    -D TPL_ENABLE_Pthread:BOOL=ON \
    -D TPL_ENABLE_BLAS:BOOL=ON \
    -D BLAS_LIBRARY_DIRS:PATH=/blas_library_path/ \
    -D BLAS_LIBRARY_NAMES:STRING="blas_library_name" \
    -D TPL_ENABLE_LAPACK:BOOL=ON \
    -D LAPACK_LIBRARY_DIRS:PATH=/lapack_library_path/ \
    -D LAPACK_LIBRARY_NAMES:STRING="lapack_library_name" \
    -D TPL_ENABLE_MPI:BOOL=ON \
    -D TPL_ENABLE_ParMETIS:BOOL=ON \
    -D ParMETIS_LIBRARY_DIRS:PATH=/parmetis_library_path/ \
    -D METIS_LIBRARY_DIRS:PATH=/metis_library_path/ \
    -D TPL_ENABLE_HDF5:BOOL=ON \
    -D MPI_BASE_DIR:PATH=/mpi_library_path/ \
    -D MPI_BIN_DIR:PATH=/mpi_binary_path/ \
    -D CMAKE_INSTALL_PREFIX:PATH=./ \
    $EXTRA_ARGS \
    /trilinos_source_path/
\end{verbatim}

Simply modify the paths and names of libraries according to your particular configuration and run the shell script.
For example, instead of \verb!lapack_library_name! you should type the name of your lapack library without the
\verb!lib! prefix and the \verb!.a! suffix. The prefix and suffix are automatically added by CMake.
After the configuration is done, just type
\begin{verbatim}
make
\end{verbatim}
that  will compile the static files and further
\begin{verbatim}
make install
\end{verbatim}
that will create and install the library files
in two subdirectories \verb|lib| and \verb|include|, where
it will respectively pack the objects files into library files (.a and .la files)
and copy the include files ( .h or .hpp files ).

The trilinos library is now installed in the build directory you created.

\subsection{Compilation from git}
\label{sec:compile-cvs}
You need first to have an account on \url{http://cmcsforge.epfl.ch} and
be part of the \lifev project, see~\ref{sec:software-management}.

First, you need to checkout \lifev. \verb!git! has
been configured to use \ixv{ssh} and your \verb!ssh! keys to
access the repository via \verb!ssh! without entering your password. When your ssh agent is properly configured,
send your public key to the local administrator, such that it can be included in
the gitolite configuration.
Then you will be able to access the
repositories without password.

It is now time to download and compile the code.
Just type
\begin{verbatim}
git clone git@cmcsforge.epfl.ch:lifev.git lifev
\end{verbatim}
where developername is your account name on cmcsforge. Place yourself in the new directory

\begin{verbatim}
cd lifev
\end{verbatim}
and type
\begin{verbatim}
git clone git@cmcsforge.epfl.ch:lifev.git cmake
\end{verbatim}



Second, you must make a build directory apart from the
lifev sources directory, for example in your home you can have a
\verb!lib! directory with a \verb!lifev! subdirectory and further
an optimized version subdirectory \verb!opt! or the debugging mode
subdirectory \verb!debug!, or something similar according to your own taste.


Third, you have to execute the
\verb+do-configure+ shell script in the \verb!opt! directory.
It will automatically check the availability of the needed components
for \lifev compilation :


\begin{verbatim}
cmake \
    -D CMAKE_BUILD_TYPE:STRING=RELEASE \
    \
    -D TPL_ENABLE_MPI:BOOL=ON \
    \
    -D ParMETIS_INCLUDE_DIRS:PATH=/parmetis_include_path/ \
    -D ParMETIS_LIBRARY_DIRS:PATH=/parmetis_library_path/ \
    \
    -D TPL_ENABLE_UMFPACK:BOOL=ON \
    \
    -D TPL_ENABLE_BLAS:BOOL=ON \
    -D BLAS_LIBRARY_DIRS:PATH=/blas_library_path/ \
    -D BLAS_LIBRARY_NAMES:STRING="blas_library_name" \
    -D TPL_ENABLE_LAPACK:BOOL=ON \
    -D LAPACK_LIBRARY_DIRS:PATH=/lapack_library_path/ \
    -D LAPACK_LIBRARY_NAMES:STRING="lapack_library_name" \
    -D TPL_ENABLE_Boost:BOOL=ON \
    -D TPL_Boost_INCLUDE_DIRS:STRING=/boost_include_path/ \
    \
    -D TPL_ENABLE_Trilinos:STRING=ON \
    -D Trilinos_INCLUDE_DIRS:PATH=/trilinos_include_path/ \
    -D Trilinos_LIBRARY_DIRS:PATH=/trilinos_library_path/ \
    \
    -D LifeV_VERBOSE_CONFIGURE:BOOL=OFF \
    -D CMAKE_VERBOSE_MAKEFILE:BOOL=ON \
    \
    -D LifeV_ENABLE_STRONG_CXX_COMPILE_WARNINGS:BOOL=OFF \
    \
    -D LifeV_ENABLE_ALL_PACKAGES:BOOL=ON \
    -D LifeV_ENABLE_TESTS:BOOL=ON \
    -D LifeV_ENABLE_EXAMPLES:BOOL=ON \
    -D CMAKE_INSTALL_PREFIX:PATH=$MY_LIB_DIR \
    $* \
    $BASE_DIR/lifev
\end{verbatim}

Do the same in the \verb!debug! directory, replacing the first line by  \begin{verbatim}    -D CMAKE_BUILD_TYPE:STRING=DEBUG\  \end{verbatim}


%\begin{verbatim}
%./configure \
%    --with-trilinos-lib=/your_trilinos_library_path/ \
%    --with-trilinos-include=/your_trilinos_include_path/ \
%    --enable-opt \
%    --with-mpi=/usr/lib/openmpi \
%    --with-mpi-lib=/usr/lib/openmpi/lib \
%    --with-mpi-include=/usr/include/openmpi \
%    --with-parmetis-include=/your_parmetis_include_path/ \
%    --with-parmetis-lib=/your_parmetis_library_path/ \
%    --with-umfpack-include=/your_umfpack_include_path/ \
%    --with-umfpack-lib=/your_umfpack_library_path/ \
%    --with-amd-include=/your_amd_include_path/ \
%    --with-amd-lib=/your_amd_library_path/ \
%    --with-hdf5=/usr \
%    --prefix=/your_lifeV_library_install_path/
%    CXXFLAGS ="-O2"
%\end{verbatim}

%\noindent to configure with optimized libraries, or change the line \verb!--enable-opt! to
%\verb!--enable-debug! and \verb!CXXFLAGS="-g2 -O0"! to build the libraries
%in debugging mode.
%
%Remember to specify the
%path to your build directory (external to the sources directory) with the flag \verb!--prefix!.
%
%Change the path and name of mpi, mpi library and include if yours is different.
%You can check for additional flags for mpi typing \verb!mpicc -show!.


\noindent Finally, you just have to use \ixv{make} to compile \lifev libraries and documentation.
Enter
\begin{verbatim}
make -j 4
make install
\end{verbatim}

\noindent Be careful because \verb!do-configure! will fail if you have already compiled
\lifev in the source directory. Therefore is not a good idea to build in the sources.

\subsection{Compilation from Official Distribution}
\label{sec:comp-from-offic}
The \lifev project provides releases, they are named using the following convention
\begin{center}
\verb!lifev-x.y.z.tar.gz!
\end{center}

Here is what you have to do:

\begin{enumerate}
\item download \lifev release \verb!lifev-x.y.z.tar.gz!
\item unpack it
\begin{verbatim}
tar -xzf lifev-x.y.z.tar.gz
\end{verbatim}
\item configure it following the instructions of the previous section,
\item compile and install it
\begin{verbatim}
make -j 4
make install
\end{verbatim}
\end{enumerate}


\subsection{Compiling Testsuites}

\noindent \lifev comes with testsuites covering a lot of features. They are located in different directories, mainly depending on the physical or technical aspects they are concerned with. For example, you can find a number of tests in the \verb+core+
directory (\verb+lifev/lifev/core/testsuite+) but \verb+darcy, fsi, navier_stokes, structure+ are other directories where you can find tests.%It is located in the directory \verb+testsuite+
%\begin{verbatim}
%|-- data
%|-- test_bdf
%|-- test_darcy
%|-- test_essentialbc
%|-- test_fe
%|-- test_fsi_newton
%|-- test_fsi_picard
%|-- test_linearelasticity
%|-- test_matrix
%|-- test_mesh
%|-- test_robin
%|-- test_naturalbc
%|-- test_ns_bdf
%|-- test_ns_cyl
%|-- test_ns_sstress
%|-- test_p2
%|-- test_postproc
%`-- test_q1
%\end{verbatim}

All these tests are automatically compiled once you have installed \lifev. To run them just type
\begin{verbatim}
make test
\end{verbatim}



%\noindent In order to compile a testsuite, you need the following steps
%\begin{verbatim}
%cd <lifev-directory>/testsuite
%make check
%\end{verbatim}
%where the lifev-directory is the directory where you unpacked \lifev.
%
%\noindent If you just want to compile a specific test, say \verb+test_darcy+
%\begin{verbatim}
%cd lifev-directory/testsuite/data
%make check
%cd lifev-directory/testsuite/test_darcy
%make test_darcy OR make check
%\end{verbatim}

%
%%%%%%%%%%%%% Some Settings for emacs and auc-TeX
% Local Variables:
% TeX-master: "lifev-dev"
% TeX-command-default: "PDFLaTeX"
% TeX-parse-self: t
% TeX-auto-save: t
% TeX-auto-regexp-list: TeX-auto-full-regexp-list
% eval: (ispell-change-dictionary "american")
% End:
%
