\documentclass{article}

\begin{document}

%======================================================================
\section{The mesh}
A mesh is initialized by two kinds of data (provided by a mesh generator)~: 

\begin{itemize}
\item a set of points (\texttt{Point}). A \textit{point} is a purely
  geometrical data, it is essentialy defined by its coordinates.
\item several sets of geometrical elements (\texttt{GeoElem}), 
namely the connectivity. Each element set defines a \texttt{Region}. 
\end{itemize}

As soon as the interpolation is chosen (namely a reference finite
element) another set must be created by the code~:

\begin{itemize}
\item a set of nodes. A node is directly related to the degrees of
  freedom of the finite elements. A node can coincides with a point but
  it is not mandatory.
\end{itemize}

Other sets may be created~:
\begin{itemize}
\item a set of faces
\item a set of edges
\end{itemize}

\subsection{Region}

\subsubsection{Description}

A \texttt{Region} is a set of geometrical elements.  It may be also
useful that a region contains a list of the nodes (although this
information is redundant since the nodes are included in the elements).

For instance, in a fluid/structure problem consisting in the flow of a 
fluid into a pipe one could have one region for the fluid, 
one for the structure, one for the interface between the fluid and the
pipe, one at the inlet, one at the outlet.

It might be necessary to distinguish boundary regions and 
``volume'' regions. But I am not convinced that it is mandatory.

Typically, we will iterate on the geometrical elements of the region
to build a matrix (or a part of it)
\begin{itemize}
\item on ``volume'' region, we build ``volume'' operator
(stiffness, mass, source)
\item on boundary region, we build natural (Neumann) boundary conditions.
\end{itemize}

Essential boundary conditions are imposed by iterating on the \textit{nodes} of
the region.

\subsection{Geometrical elements}
At the beginning (as soon as the mesh is loaded) the geometrical 
element contain the \textit{points}' numbers. After the reference finite 
element has been chosen, it must also contain the \textit{nodes}' numbers. 

%======================================================================
\section{Field}

\subsection{Description}
A \texttt{Field} is a high-level object that should be able to describe 
the discret counterpart of the following mathematical entities~:
\begin{itemize}
\item the unknown of a pde (vectorial or scalar).
\item any analytical functions (used for example to impose boundary
  conditions, to define source terms, exact known solutions, 
  non-homogeneous physical parameters,...).
\end{itemize}

To define a \texttt{Field},it should be sufficient to provide a
\texttt{Region} $\Omega$ a number of components.

For example, in a typical fluid structure interaction problem, one could
define three \texttt{Field}s for the unknown~: the velocity (3
components, region ``fluid''), the pressure (1
component, region ``fluid''), the displacement (3
components, region ``solid'').

\subsection{Methods}
What do we want to be able to do with a Field ? 
\begin{itemize}
\item compute their norm in $l^\infty$, $L^2(\Omega)$, $H^1(\Omega)$ and
  $H(div)$, $H(rot)$ for the vectorial fields. 
\item initialize the fields values with an analytical function, or with
  the values of another field.
\item the field values are \emph{a priori} defined at the
  \emph{nodes} of the region. It may be useful to be able to extract
  their values at the \emph{points} (for post processing purpose for example).
\end{itemize}

\subsection{Interface}
\textbf{It needs~:}
\begin{itemize}
\item a \texttt{Region}.
\end{itemize}

\textbf{It is necessary to~:}
\begin{itemize}
\item define a linear problem
\end{itemize}

%======================================================================
\section{The finite elements}


%======================================================================
\section{Miscellaneous}

\subsection{Boundary conditions}



\end{document}

% $Id: jfg.tex,v 1.2 2004-02-08 08:54:49 prudhomm Exp $ \n