\chapter{HOWTO}

\section{Geometrical mappings}

In file \texttt{geoMap.h}:
 
GeoMap         : analytical mapping for parametric Lagrangian map
 
GeoMapQuadRule : derived from GeoMap when a quadrature rule is given 
  
GeoMap is the base class, two kinds of classes can be derived from it:

\begin{itemize}
\item GeoMapQuadRule : when a quadrature rule is used
\item various classes when an "exact" integration is used
\end{itemize}
 

\section{HOWTO add new basis functions}

The basis functions may be used  to define the geometrical mappings and
the finite element shape functions. A lot of basis functions are already
implemented in the file \verb#basisFct.h#. To create a new set of basis 
functions you must create a class which derives from the class 
{\verb BasisFct_Base<Shape,nbFct>}, where Shape is either Line or 
Quad or Triangle or Tetra or Hexa, and nbFct is the number of basis functions.
Your class has just to contain a constructor which fills the arrays fct[i] 
($0\leq i < nbFct$), and fctDer[i][icoor], ($0\leq i < nbFct$ and 
$0<icoor<Shape::S_nDimensions$). These arrays contain pointers to the basis
functions and their derivatives. These function are staightforwardly
coded as global function which take the coordinates $x,y,z$ as 
arguments and return the value of the function (see examples in the file   
\texttt{basisFct.h} ).

\section{HOWTO add a new quadrature rule}

The way is very similar to the one described for the basis
functions. You have just to derive a new class from the class
{\verb QuadRule<Shape,nbQuadPt>}. See the file \texttt{quadRule.h}

\section{HOWTO add a new element operator}

By ``element operator'', we mean element matrix corresponding to
a \ixs{differential}{operator} \ix{operator}. For example, the element \ixs{stiffness}{operator} or \ixs{mass}{operator}
matrices.

$\bullet${\bf First case}: you use a quadrature rule. Just add your 
operator in the file \texttt{elemOper.h} by copying and modifying an
existent one (e.g. stiff or mass). By this way, the new operator will 
work with arbitrary general finite elements. 

$\bullet${\bf Second case}: if you want to code ``hardly'' (without
integration rule) the element matrix corresponding to your operator for
a given finite element, you must create a ``specialized'' finite
element. This may be much more efficient, but the new operator will work
only with this element.

\section{HOWTO build a new finite element}

In \texttt{feBase.h}: create a base class for the new element (copy,
paste and modify an existant one...), say {\verb FE_My_Tetra_Base}. The
following depends on what you want to do.

$\bullet${\bf First case}: you want to use a quadrature rule. The
advantage: you will straighforwardly inherit all the work already done
for the other finite elements (in particular the operators defined in
\texttt{elemOper.h}). The drawback: if your finite element has a low order,
the comptationnal cost may be  more expensive than with a
``specialized integration''.

- have a look at the end of the file \texttt{geoMap.h}, you will
  find many geometrical mapping together with integration rules. Choose
  one, say \verb#GeoMap_Linear_Tetra_4pt#. If you don't like the
  geometrical mappings already defined, you have to create a new one.
  Remember that a geomapQR class is the ``tensorial product'' (template
  in fact...) of a set of basis function (in this example,
  \verb#BasisFct_P1_3D#, see \texttt{basisFct.h}), a geometric element (here,
  GeoElement3D<LinearTetra>, see \texttt{basisElSh.h}) and a quadrature
  rule (here, \verb#QuadRule_Tetra_4pt#, see \texttt{quadRule.h}).

- in \texttt{finiteEleQR.h}: create a typedef correponding to the
  ``product'' of your base finite element and the geomap:
\begin{verbatim}
typedef FiniteEleQR<FE_My_Tetra_Base,GeoMap_Linear_Tetra_4pt> FE_My_Tetra_4pt;
\end{verbatim}
  
  $\bullet${\bf Second case}: you don't want to use a quadrature rule.
  Advantage: probably more efficient. Drawback: you have to reimplement
  your own element operators. Create a \texttt{MyGeoMap} class where you
  can store some geometrical stuffs, and a class \texttt{MyFE} where you
  store what you want accordingly to you finite element. The job now
  consists in developping the \emph{specialized} operators. 
It should look like
\begin{verbatim}
template<>
void stiff<MyFE,MyGeoMap>(Real coef,ElemMat& elmat,const MyFE& fe,
const MyGeoMap& geo,int iblock=0,int jblock=0)
...

\end{verbatim}
Thanks to the specialization, as soon as you adopt the same interface
for your operator than those of \texttt{elemOper.h}, a code
that works with a ``general'' FE (i.e. with a quadrature rule)
will work without any change with your new element.

You can have a look at \verb#fe_p1_tetra_exact.h#. But I emphasize that 
this is just a test, and it should probably be improved (in particular,
the geometrical quantities stored in the geomap class...).


%
%%%%%%%%%%%%% Some Settings for emacs and auc-TeX
% Local Variables:
% TeX-master: "lifev-dev"
% TeX-command-default: "PDFLaTeX"
% TeX-parse-self: t
% TeX-auto-save: t
% x-symbol-8bits: nil
% TeX-auto-regexp-list: TeX-auto-full-regexp-list
% eval: (ispell-change-dictionary "american")
% End:
%
